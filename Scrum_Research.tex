\documentclass[11pt]{amsart}
\linespread{1.6}
\usepackage{fullpage}
\usepackage{color}

\title{A case study in the use of scrum for managing academic research projects}

\author{D.~Z.~Turner}
\address{Correspondence to: Dr. Daniel Z. Turner, Department of Civil Engineering, 
  Stellenbosch University, Private Bag X1 Matieland 7602, South Africa. \emph{E-mail address:} {\ttfamily dzturner@sun.ac.za}, \emph{Phone: } +27-21-808-4434}

\author{A.~A.~Lorber}
\address{Dr. Alfred A. Lorber, Computational Thermal \& Fluid Mechanics, 
  Sandia National Laboratories, PO Box 5800 MS 0836, Albuquerque, New Nexico 87185. \emph{E-mail address:} {\ttfamily aalorbe@sandia.gov}, \emph{Phone: } +1-505-845-9712}
% 
\date{\today}
\begin{document}

%*************************;
%  Abstract of the paper  ;
%*************************;
\begin{abstract}
The following work presents a case study in the application of scrum (agile) project management to an academic research context. A number of metrics are monitored to evaluate the effectiveness of the modified scrum strategy developed herein. The metrics include a comparison of resource investment vs. stakeholder satisfaction, avoidance of early project lulls and late project spikes in productivity, and an assessment of how well project goals were accomplished.
{\color{red}Add a quick summary of the conclusions once completed}
\end{abstract}

%************;
%  Keywords  ;
%************;
\keywords{scrum; agile; academic research context}

\maketitle

\section{INTRODUCTION AND MOTIVATION}
Scrum (agile) project management has recently transformed the software development industry due to its ability to adapt to changing project requirements, the iterative nature of its execution, and its emphasis on teamwork. All of these attributes address critical weaknesses in the current way most academic research projects are conducted. The features scrum has to offer motivate this present work which tailors a new scrum-based methodology for an academic research context.
\begin{itemize}
\item Overview of the research project
\item How this has potential for a better way to manage research projects
\end{itemize}
\section{AGILE PROJECT MANAGEMENT METHODOLOGY}
In this section we give a brief overview of the essential elements of a scrum based project management methodology and describe how this methodology was adapted for an academic research context.
\begin{itemize}
\item What is involved in scrum
\item How we modified this for the research project
\end{itemize}
\section{CASE STUDY IN ACADEMIC RESEARCH}
The methodology outlined above was used to manage a research project that involved 
\begin{itemize}
\item More details on project
\item Description of the metrics and how they were collected
\end{itemize}
\section{OUTCOMES AND DISCUSSION}
In this section we present the results of a number of metrics that were monitored in order to gauge the efficacy of the methodology presented above.
\begin{itemize}
\item Plots of the data
\item Discussion of results for each plot
\item Reflections on what went well, what could be improved
\item How did we improve based on the outcomes, iteration
\item Testimonials
\end{itemize}
\section{CONCLUSIONS}
Given the outcomes discussed above, the following conclusions can be drawn:
\begin{itemize}
\item Conclusions go here
\item What do these results suggest about this process vs. the traditional approach
\end{itemize}
\section{RECOMMENDATIONS AND BEST PRACTICES}
Based on the results and conclusions above, the following guidelines should be followed when implementing a similar application of scrum to a research project.

\begin{itemize}
\item Best practices go here.
\end{itemize}

%\bibliographystyle{unsrt}
%\bibliography{../Master_References/Books,../Master_References/Master_References}

\section{APPENDIX A: CASE STUDY EPIC AND STORIES}
{\color{red}This may or may not be included in the final draft, but I put it here so we can iterate in one document. I'll add these once I have had a chance to think about them a bit.}

\end{document}
